\documentclass[a4paper,12pt]{article}
\usepackage[utf8]{inputenc}
\usepackage[T1]{fontenc}
\usepackage[italian]{babel}
\usepackage{amsmath}
\usepackage{graphicx}
\usepackage{lmodern}
\usepackage{hyperref}

\title{\textbf{Progetto del Corso di Programmazione a Oggetti}\\\large MinecraftManager}
\author{Nicola Simionato}
\date{\today}

\begin{document}

\maketitle

\section{Introduzione}

\textit{MinecraftModManager} è un software sviluppato con l'obiettivo di facilitare lo sviluppo di mod per \textit{Minecraft}.  
Sebbene la struttura interna del gioco a cui si fa riferimento sia molto più complessa e articolata rispetto all'implementazione qui proposta, il progetto si rivela comunque utile in diversi contesti.

In particolare, una delle funzionalità implementate consente la generazione automatica di un file \texttt{.json} utilizzato nelle mod per la gestione delle traduzioni dei nomi di oggetti, blocchi, materiali e altri elementi di gioco.

\end{document}
